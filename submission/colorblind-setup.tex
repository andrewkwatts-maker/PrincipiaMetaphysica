% ============================================
% COLOR-BLIND FRIENDLY COLOR DEFINITIONS
% ============================================
% For Principia Metaphysica APS submission
%
% Based on Paul Tol's color schemes for colorblindness accessibility
% Reference: https://personal.sron.nl/~pault/
%
% These colors are distinguishable for:
% - Deuteranopia (red-green colorblindness, most common)
% - Protanopia (red-green colorblindness, less common)
% - Tritanopia (blue-yellow colorblindness, rare)
% - Normal color vision
%
% IMPORTANT: Always combine color with line style/markers/patterns
% ============================================

% Required packages
\usepackage{xcolor}
\usepackage{tikz}
\usetikzlibrary{patterns}

% ============================================
% PRIMARY PALETTE (Paul Tol's Bright Scheme)
% ============================================
% Use these for most figures. They are maximally distinct.

\definecolor{cbBlue}{RGB}{0,114,178}        % Blue - primary color
\definecolor{cbOrange}{RGB}{213,94,0}       % Orange/Vermillion - secondary color
\definecolor{cbGreen}{RGB}{0,158,115}       % Bluish green - tertiary color
\definecolor{cbYellow}{RGB}{240,228,66}     % Yellow - caution color
\definecolor{cbPurple}{RGB}{204,121,167}    % Reddish purple
\definecolor{cbRed}{RGB}{230,159,0}         % Orange-red (NOT true red!)

% Note: "cbRed" is actually orange-red, safe for colorblind users
% Never use pure red (#FF0000) or pure green (#00FF00) together!

% ============================================
% EXTENDED PALETTE (Additional Colors)
% ============================================
% For complex figures with many data series

\definecolor{cbDarkBlue}{RGB}{0,68,136}     % Darker blue variant
\definecolor{cbLightBlue}{RGB}{86,180,233}  % Lighter blue variant
\definecolor{cbDarkGreen}{RGB}{0,109,44}    % Darker green variant
\definecolor{cbLightGreen}{RGB}{102,194,165} % Lighter green variant
\definecolor{cbDarkOrange}{RGB}{179,88,6}   % Darker orange variant
\definecolor{cbLightOrange}{RGB}{253,184,99} % Lighter orange variant

% ============================================
% GRAYSCALE (For shading/backgrounds)
% ============================================

\definecolor{cbGray10}{gray}{0.1}   % Very dark gray (10% white)
\definecolor{cbGray30}{gray}{0.3}   % Dark gray
\definecolor{cbGray50}{gray}{0.5}   % Medium gray
\definecolor{cbGray70}{gray}{0.7}   % Light gray
\definecolor{cbGray90}{gray}{0.9}   % Very light gray (90% white)

% ============================================
% USAGE RECOMMENDATIONS
% ============================================
%
% TWO COLORS: Use cbBlue + cbOrange
%   Example: Theory vs. Experiment, Prediction vs. Data
%
% THREE COLORS: Use cbBlue + cbOrange + cbGreen
%   Example: Three different models or datasets
%
% FOUR+ COLORS: Add cbYellow, cbPurple, cbRed in that order
%   Always supplement with line styles and markers!
%
% BACKGROUNDS: Use cbGray90 (very light) or cbGray70 (light)
%   Ensure 4.5:1 contrast ratio with foreground colors
%
% EMPHASIS: Use cbRed (orange-red) sparingly for highlights
%   Example: Marking exact matches, critical points
%
% ============================================

% ============================================
% TIKZ LINE STYLE DEFINITIONS
% ============================================
% Always use varied line styles in addition to colors!

\tikzset{
  % Basic line styles (use with colors)
  solid/.style={solid, line width=1.2pt},
  dashed/.style={dashed, line width=1.2pt},
  dotted/.style={dotted, line width=1.5pt},  % Thicker for visibility
  dashdot/.style={dash pattern=on 4pt off 2pt on 1pt off 2pt, line width=1.2pt},
  dashdotdot/.style={dash pattern=on 4pt off 2pt on 1pt off 2pt on 1pt off 2pt, line width=1.2pt},

  % Thick variants for emphasis
  thicksolid/.style={solid, line width=2pt},
  thickdashed/.style={dashed, line width=2pt},
  thickdotted/.style={dotted, line width=2.5pt},

  % Ultra-thick for key results
  ultrathick/.style={line width=3pt},
}

% ============================================
% TIKZ MARKER DEFINITIONS
% ============================================
% Use varied markers in addition to colors and line styles!

\tikzset{
  % Basic markers
  markCircle/.style={mark=o, mark size=2.5pt, mark options={solid}},
  markSquare/.style={mark=square, mark size=2.5pt, mark options={solid}},
  markTriangle/.style={mark=triangle, mark size=3pt, mark options={solid}},
  markDiamond/.style={mark=diamond, mark size=3pt, mark options={solid}},
  markStar/.style={mark=star, mark size=3pt, mark options={solid}},
  markPentagon/.style={mark=pentagon, mark size=3pt, mark options={solid}},

  % Filled markers
  markCircleFilled/.style={mark=*, mark size=2.5pt},
  markSquareFilled/.style={mark=square*, mark size=2.5pt},
  markTriangleFilled/.style={mark=triangle*, mark size=3pt},
  markDiamondFilled/.style={mark=diamond*, mark size=3pt},

  % Larger markers for emphasis
  markCircleLarge/.style={mark=o, mark size=3.5pt, mark options={solid, thick}},
  markSquareLarge/.style={mark=square, mark size=3.5pt, mark options={solid, thick}},
}

% ============================================
% TIKZ PATTERN DEFINITIONS
% ============================================
% Use for filled regions (bar charts, area plots)

\tikzset{
  % Hatching patterns (combine with colors at ~50% opacity)
  hatchHorizontal/.style={pattern=horizontal lines, pattern color=black!50},
  hatchVertical/.style={pattern=vertical lines, pattern color=black!50},
  hatchDiagonalUp/.style={pattern=north east lines, pattern color=black!50},
  hatchDiagonalDown/.style={pattern=north west lines, pattern color=black!50},
  hatchGrid/.style={pattern=grid, pattern color=black!50},
  hatchDots/.style={pattern=dots, pattern color=black!50},

  % Colored patterns (for area fills)
  fillBlue/.style={fill=cbBlue!20},
  fillOrange/.style={fill=cbOrange!20},
  fillGreen/.style={fill=cbGreen!20},
  fillYellow/.style={fill=cbYellow!30},  % Lighter to avoid overwhelming
}

% ============================================
% COMBINED STYLE PRESETS
% ============================================
% Ready-to-use combinations for common scenarios

\tikzset{
  % Line plot styles (color + line style + marker)
  linePlot1/.style={cbBlue, solid, markCircle},
  linePlot2/.style={cbOrange, dashed, markSquare},
  linePlot3/.style={cbGreen, dotted, markTriangle},
  linePlot4/.style={cbPurple, dashdot, markDiamond},
  linePlot5/.style={cbRed, dashdotdot, markStar},

  % Bar chart styles (color + pattern)
  bar1/.style={fill=cbBlue, draw=black},
  bar2/.style={fill=cbOrange, draw=black},
  bar3/.style={fill=cbGreen, draw=black},
  bar1Hatched/.style={fill=cbBlue!50, hatchDiagonalUp, draw=black},
  bar2Hatched/.style={fill=cbOrange!50, hatchDiagonalDown, draw=black},

  % Emphasis styles
  highlight/.style={cbRed, line width=2pt},
  exactMatch/.style={fill=cbRed, draw=black, line width=1.5pt},
  withinOneSigma/.style={fill=cbBlue, draw=black},
  withinTwoSigma/.style={fill=cbLightBlue, draw=black},
}

% ============================================
% USAGE EXAMPLES
% ============================================

% Example 1: Simple line plot with two data series
% \begin{tikzpicture}
%   \draw[linePlot1] (0,0) -- (1,1) -- (2,1.5) -- (3,2);
%   \draw[linePlot2] (0,0.5) -- (1,0.8) -- (2,1.1) -- (3,1.3);
% \end{tikzpicture}

% Example 2: Bar chart with three categories
% \begin{tikzpicture}
%   \filldraw[bar1] (0,0) rectangle (0.8,2);
%   \filldraw[bar2] (1,0) rectangle (1.8,3);
%   \filldraw[bar3] (2,0) rectangle (2.8,1.5);
% \end{tikzpicture}

% Example 3: Scatter plot with markers
% \begin{tikzpicture}
%   \foreach \x/\y in {0/0, 0.5/0.8, 1/1.2, 1.5/1.8, 2/2.1}
%     \draw[cbBlue, markCircleFilled] (\x,\y) node {};
%   \foreach \x/\y in {0/0.5, 0.5/0.6, 1/1.0, 1.5/1.3, 2/1.6}
%     \draw[cbOrange, markSquareFilled] (\x,\y) node {};
% \end{tikzpicture}

% Example 4: Filled region under curve
% \begin{tikzpicture}
%   \fill[fillBlue] (0,0) -- (0,1) -- (1,1.5) -- (2,1.8) -- (3,2) -- (3,0) -- cycle;
%   \draw[cbBlue, thicksolid] (0,1) -- (1,1.5) -- (2,1.8) -- (3,2);
% \end{tikzpicture}

% ============================================
% PGFPLOTS COMPATIBILITY
% ============================================
% For use with pgfplots package

% Define cycle list for automatic color/style cycling
\pgfplotscreateplotcyclelist{colorblind}{%
  {cbBlue, solid, mark=o, mark options={solid, fill=cbBlue}},
  {cbOrange, dashed, mark=square, mark options={solid, fill=cbOrange}},
  {cbGreen, dotted, mark=triangle, mark options={solid, fill=cbGreen}},
  {cbPurple, dashdot, mark=diamond, mark options={solid, fill=cbPurple}},
  {cbRed, dashdotdot, mark=star, mark options={solid, fill=cbRed}},
}

% Usage in pgfplots:
% \begin{axis}[cycle list name=colorblind]
%   \addplot coordinates {(0,0) (1,1) (2,2)};
%   \addplot coordinates {(0,0.5) (1,1.2) (2,1.8)};
% \end{axis}

% ============================================
% COLOR TESTING
% ============================================
% Uncomment to generate color swatch page for testing

% \newpage
% \section*{Colorblind-Safe Color Palette}
% \begin{tikzpicture}
%   % Primary palette
%   \foreach \color/\name/\y in {
%     cbBlue/Blue/0,
%     cbOrange/Orange/1,
%     cbGreen/Green/2,
%     cbYellow/Yellow/3,
%     cbPurple/Purple/4,
%     cbRed/Orange-Red/5
%   } {
%     \filldraw[\color] (0,-\y) rectangle (2,-\y-0.8);
%     \node[right] at (2.2,-\y-0.4) {\name};
%   }
% \end{tikzpicture}

% ============================================
% ACCESSIBILITY NOTES
% ============================================
%
% 1. ALWAYS combine color with line style and/or markers
%    - Color alone is insufficient for colorblind readers
%    - Line style ensures grayscale printing is readable
%
% 2. Test your figures in grayscale
%    - Print preview in black & white
%    - All lines/regions should remain distinguishable
%
% 3. Use colorblindness simulators
%    - Coblis: https://www.color-blindness.com/coblis-color-blindness-simulator/
%    - Color Oracle: https://colororacle.org/
%
% 4. Minimum contrast ratios (WCAG AA standard)
%    - Text: 4.5:1 for normal text, 3:1 for large text
%    - Graphics: 3:1 for meaningful graphics
%    - Use WebAIM contrast checker: https://webaim.org/resources/contrastchecker/
%
% 5. Caption accessibility
%    - Explicitly describe colors in captions
%    - "Solid blue line shows theory, dashed orange line shows experiment"
%    - Don't rely on readers perceiving colors correctly
%
% 6. Legend placement
%    - Place legends inside plot area when possible
%    - Ensure legend labels match line styles exactly
%    - Use \addlegendentry in pgfplots for consistency
%
% ============================================
% END OF COLOR SETUP
% ============================================
