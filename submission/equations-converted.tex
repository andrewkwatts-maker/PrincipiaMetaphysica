% ============================================
% PRINCIPIA METAPHYSICA - KEY EQUATIONS
% Extracted from principia-metaphysica-paper.html
% Converted to LaTeX for REVTeX submission
% Generated: 2025-12-06
% ============================================

% ============================================
% SECTION 2.1: GEOMETRIC FRAMEWORK
% ============================================

% Master 26D Action
\begin{equation}
\label{eq:master-action-26d}
S = \int d^{26}X \sqrt{-G} \left[ R + \bar{\Psi}_P \left(i\Gamma^M D_M - m\right) \Psi_P + \mathcal{L}_{\text{Sp}(2,\mathbb{R})} \right]
\end{equation}

% 26D spacetime decomposition
\begin{equation}
\label{eq:26d-decomposition}
M^{26} = M^{(4,2)} \times K_{\text{Pneuma}} \times \tilde{K}_{\text{Pneuma}}
\end{equation}

% Sp(2,R) gauge constraints (eliminate ghosts from 2nd time)
\begin{equation}
\label{eq:sp2r-constraints}
X^2 = 0, \quad X \cdot P = 0, \quad P^2 + M^2 = 0
\end{equation}

% Dimensional reduction chain
\begin{equation}
\label{eq:dimensional-reduction}
M^{26}(24,2) \xrightarrow{\text{Sp}(2,\mathbb{R})} M^{13}(12,1) \xrightarrow{G_2} M^6(5,1)
\end{equation}

% Two-time combination
\begin{equation}
\label{eq:two-time-combination}
t_{\text{total}} = t_{\text{therm}} + \beta \cdot t_{\text{ortho}}, \quad \beta = \cos(\theta_{\text{mirror}})
\end{equation}

% F(R,T,τ) gravity action
\begin{equation}
\label{eq:frt-gravity-action}
S_{\text{grav}} = \int d^4x \sqrt{|g|} \, F(R, T, \tau)
\end{equation}

% F(R,T,τ) form
\begin{equation}
\label{eq:frt-form}
F(R, T, \tau) = R + f(T) + \lambda_\tau \tau + \Lambda(\tau)
\end{equation}

% Condensate gap equation
\begin{equation}
\label{eq:condensate-gap}
\Delta = \frac{\lambda v}{1 + g \cdot t_{\text{ortho}}/E_F}
\end{equation}

% Clifford algebra in 26D
\begin{equation}
\label{eq:clifford-26d}
\{\Gamma^M, \Gamma^N\} = 2G^{MN} I_{8192}, \quad \dim = 2^{13} = 8192
\end{equation}

% Clifford algebra in effective 13D
\begin{equation}
\label{eq:clifford-13d}
\{\Gamma^M_{\text{eff}}, \Gamma^N_{\text{eff}}\} = 2\eta^{MN} I_{64}, \quad \dim = 2^6 = 64
\end{equation}

% Anomaly cancellation (central charge)
\begin{align}
\label{eq:central-charge-26d}
c_{\text{total}} &= c_{\text{matter}} + c_{\text{ghost}} = 26 + (-26 + 2) = 2 \\
c_{\text{matter,eff}} &= 24 \quad \Rightarrow \quad c_{\text{total}} = 24 - 26 + 2 = 0
\label{eq:central-charge-effective}
\end{align}

% 13D effective spacetime
\begin{equation}
\label{eq:13d-effective}
M^{13}_{\text{eff}} = M^6_{\text{bulk}} \times G_2^{\text{Pneuma}}
\end{equation}

% Brane structure
\begin{equation}
\label{eq:brane-decomposition}
M^{13}_{\text{eff}} = (B_1^3 \oplus B_2^3 \oplus B_3^3 \oplus B_4^3) \times \mathbb{R}_{t_{\text{therm}}}
\end{equation}

% Octonion-inspired dimension split
\begin{equation}
\label{eq:octonion-dimensions}
D = 13 = 1 + 4 + 8 = \dim(\mathbb{R}) + \dim(\mathbb{H}) + \dim(\mathbb{O})
\end{equation}

% Planck mass relation
\begin{equation}
\label{eq:planck-mass}
M_{\text{Pl}}^2 = M_*^{11} \cdot V_8
\end{equation}

% 4D effective Lagrangian
\begin{equation}
\label{eq:4d-lagrangian}
\mathcal{L}_{\text{4D}} = f(R, T, \tau) + \mathcal{L}_{\text{SM}} + y_{ij} \bar{f}_i f_j \phi + \mathcal{L}_{\phi_M}
\end{equation}


% ============================================
% SECTION 2.2: TOPOLOGY AND GENERATIONS
% ============================================

% Euler characteristic formula
\begin{equation}
\label{eq:euler-char-formula}
\chi_{\text{eff}} = b_0 - b_1 + b_2 - b_3 + b_4 - b_5 + b_6 - b_7
\end{equation}

% Bare Euler characteristic
\begin{equation}
\label{eq:euler-char-bare}
\chi_{\text{bare}} = 1 - 0 + 4 - 24 + 24 - 0 + 0 - 1 = 4
\end{equation}

% Flux-dressed Euler characteristic (CRITICAL PREDICTION!)
\begin{equation}
\label{eq:euler-char-effective}
\chi_{\text{eff}}(G_2) = 144 \quad \text{(from flux-dressed TCS $G_2$ topology)}
\end{equation}

% Generation count formula (CRITICAL PREDICTION!)
\begin{equation}
\label{eq:generation-count}
n_{\text{gen}} = \frac{\chi_{\text{eff}}}{48} = \frac{144}{48} = 3
\end{equation}

% M2-brane count
\begin{equation}
\label{eq:m2-brane-count}
N_{M2} = \frac{\chi_{\text{eff}}(G_2)}{24}
\end{equation}

% TCS G2 construction
\begin{equation}
\label{eq:tcs-g2}
G_2^{\text{TCS}} = (Z_+ \times_\theta Z_-) / S^1
\end{equation}

% G2 associative calibration
\begin{equation}
\label{eq:g2-calibration}
d\varphi = 0, \quad d(*\varphi) = 0
\end{equation}

% G2 associative 3-cycles
\begin{equation}
\label{eq:g2-3cycles}
G_2^{\text{Pneuma}} \supset \Sigma_1 \cup \Sigma_2 \cup \Sigma_3 \cup \Sigma_4 \quad \text{(associative 3-cycles)}
\end{equation}


% ============================================
% SECTION 2.3: GAUGE UNIFICATION
% ============================================

% I*_1 ↔ D_5 singularity (SO(10) emergence)
\begin{equation}
\label{eq:d5-singularity}
I^*_1 \leftrightarrow D_5 \quad \Rightarrow \quad \text{gauge group } \mathfrak{so}(10)
\end{equation}

% Discriminant non-vanishing (elliptic fibration)
\begin{equation}
\label{eq:discriminant-elliptic}
4f^3 + 27g^2 \neq 0 \quad \text{(generically on } S\text{)}
\end{equation}

% RG evolution of gauge couplings
\begin{equation}
\label{eq:rg-evolution}
\alpha_i^{-1}(\mu) = \alpha_i^{-1}(M_Z) - \frac{b_i}{2\pi} \ln\left(\frac{\mu}{M_Z}\right)
\end{equation}

% GUT scale and coupling (CRITICAL PREDICTION!)
\begin{equation}
\label{eq:gut-scale}
M_{\text{GUT}} \approx 2 \times 10^{16} \text{ GeV}, \quad \alpha_{\text{GUT}}^{-1} \approx 24
\end{equation}

% Torsion parameter (CRITICAL - geometrically derived!)
\begin{equation}
\label{eq:torsion-omega}
T_\omega = \ln\left(4 \sin^2\left(\frac{5\pi}{48}\right)\right) = -0.884
\end{equation}

% Log scale ratio
\begin{equation}
\label{eq:log-scale-ratio}
\ln\left(\frac{M_{\text{Pl}}}{M_{\text{GUT,base}}}\right) = 6.519
\end{equation}

% Warp scaling parameter
\begin{equation}
\label{eq:warp-parameter}
s = \frac{\log_{\text{scale}} - T_\omega}{\text{flux}_{\text{norm}}}
\end{equation}

% Warp coupling constant
\begin{equation}
\label{eq:warp-coupling}
c_{\text{warp}} = \frac{3}{22 - \nu/12} = \frac{3}{22 - 24/12} = \frac{3}{20}
\end{equation}

% GUT mass with warping (geometrically determined!)
\begin{equation}
\label{eq:gut-mass-warped}
M_{\text{GUT}} = M_{\text{GUT,base}} \times (1 + c_{\text{warp}} \times s) = 2.118 \times 10^{16} \text{ GeV}
\end{equation}

% KK threshold corrections
\begin{equation}
\label{eq:kk-threshold}
\Delta_{\text{TC}} = +\frac{k_i}{2\pi} \log\left(\frac{M_{\text{GUT}}}{M_*}\right)
\end{equation}

% Beta function with KK modes
\begin{equation}
\label{eq:beta-kk}
\beta_{\text{KK}}(g) = \beta_{\text{SM}}(g) + \frac{2}{2\pi} \alpha_i^2 \left(\frac{\mu}{M_*}\right)^2
\end{equation}


% ============================================
% SECTION 2.4: FERMION SECTOR
% ============================================

% SO(10) 16-plet
\begin{equation}
\label{eq:so10-16plet}
\mathbf{16} = (u, d, e, \nu)_L + (u^c, d^c, e^c, \nu^c)_R
\end{equation}

% Neutrino mass Lagrangian
\begin{equation}
\label{eq:neutrino-mass-lagrangian}
\mathcal{L}_{\text{mass}} = -m_D \bar{\nu}_L \nu_R - \frac{1}{2} M_R \bar{\nu}_R^c \nu_R + \text{h.c.}
\end{equation}

% Seesaw mass matrix
\begin{equation}
\label{eq:seesaw-matrix}
M = \begin{pmatrix} 0 & m_D \\ m_D & M_R \end{pmatrix}
\end{equation}

% Seesaw formula (type-I)
\begin{equation}
\label{eq:seesaw-formula}
m_\nu \approx \frac{m_D^2}{M_R}, \quad M_N \approx M_R
\end{equation}

% Neutrino mass scale estimate
\begin{equation}
\label{eq:neutrino-mass-scale}
m_\nu \sim \frac{(170 \text{ GeV})^2}{10^{16} \text{ GeV}} \sim 0.003 \text{ eV}
\end{equation}

% Yukawa coupling from wavefunction overlap
\begin{equation}
\label{eq:yukawa-overlap}
Y_{\alpha\beta\gamma} = \int \psi_\alpha \psi_\beta \varphi_\gamma \, dV_{G_2}
\end{equation}

% CKM matrix structure (from geometry)
\begin{equation}
\label{eq:ckm-structure}
C_{e\pi^0} \sim \text{Tr}\left(V_{\text{CKM}} Y_{\text{up}} V_{\text{CKM}}^\dagger \times Y_{\text{down}} \times Y_{\text{lepton}}\right)
\end{equation}

% Branching ratio formula
\begin{equation}
\label{eq:branching-ratio}
\text{BR}(i) = \frac{|C_i|^2}{\sum_j |C_j|^2}
\end{equation}


% ============================================
% SECTION 2.5: COSMOLOGY AND DARK ENERGY
% ============================================

% Tomita-Takesaki modular flow
\begin{equation}
\label{eq:modular-flow}
\rho = \frac{e^{-K}}{Z}, \quad \alpha_t(A) = e^{iKt} A e^{-iKt}
\end{equation}

% KMS condition
\begin{equation}
\label{eq:kms-condition}
\omega(AB) = \omega(B \sigma_i(A)) \quad \text{[KMS condition]}
\end{equation}

% Thermal entropy current
\begin{equation}
\label{eq:thermal-current}
J_S^\mu = s u^\mu + \frac{q^\mu}{T}
\end{equation}

% Thermal time parameter
\begin{equation}
\label{eq:thermal-time-parameter}
\alpha_T = \frac{d \ln \tau}{d \ln a} - \frac{d \ln H}{d \ln a}
\end{equation}

% Thermal time scaling
\begin{align}
\tau &= 1/\Gamma \propto a \quad \Rightarrow \quad \frac{d \ln \tau}{d \ln a} = +1 \label{eq:tau-scaling} \\
H &\propto a^{-3/2} \quad \Rightarrow \quad \frac{d \ln H}{d \ln a} = -3/2 \label{eq:hubble-scaling}
\end{align}

% Thermal parameter value
\begin{equation}
\label{eq:thermal-parameter-value}
\alpha_T(0) = \left(\frac{d \ln \tau}{d \ln a} - \frac{d \ln H}{d \ln a}\right) = (+1) - (-3/2) = 5/2
\end{equation}

% Mirror correction
\begin{equation}
\label{eq:mirror-correction}
\alpha_T = \alpha_T(0) + \delta\alpha_{\text{mirror}} = 2.5 + 0.2 = 2.7
\end{equation}

% Mashiach field potential
\begin{equation}
\label{eq:mashiach-potential}
V(\varphi) = V_0 e^{-\lambda\varphi/M_{\text{Pl}}} \left[1 + A \cos(\omega\varphi + \theta)\right]
\end{equation}

% Dark energy equation of state evolution (CRITICAL PREDICTION!)
\begin{equation}
\label{eq:dark-energy-evolution}
w(z) = w_0 \left[1 + \frac{\alpha_T}{3} \ln(1+z)\right]
\end{equation}

% w_0 from effective dimensions (CRITICAL - geometrically derived!)
\begin{equation}
\label{eq:w0-from-dimensions}
w_0 = -\frac{2(D_{\text{eff}} - 1)}{D_{\text{eff}} + 1} = -\frac{2(12.577 - 1)}{12.577 + 1} = -0.853
\end{equation}

% Mashiach field equation of motion
\begin{equation}
\label{eq:mashiach-eom}
\ddot{\varphi} + 3H\dot{\varphi} + \Gamma\dot{\varphi} + V'(\varphi) = 0
\end{equation}

% Slow-roll approximation
\begin{equation}
\label{eq:slow-roll}
(3H + \Gamma)\dot{\varphi} \approx -V'(\varphi)
\end{equation}

% w(z) with activation
\begin{equation}
\label{eq:wz-activation}
w(z) = w_0 \left[1 + \frac{\alpha_T}{3} \ln\left(\frac{1+z}{z_{\text{act}}}\right)\right]
\end{equation}

% Planck breathing mode bias
\begin{equation}
\label{eq:planck-bias}
\Delta w_0^{\text{Planck}} = -0.07 \pm 0.03
\end{equation}


% ============================================
% SECTION 2.6: PROTON DECAY
% ============================================

% Effective dimension-6 operator
\begin{equation}
\label{eq:proton-decay-operator}
\mathcal{L}_{\text{eff}} = \frac{g_{\text{GUT}}^2}{M_X^2} \epsilon_{ijk} (u_i^c d_j)(Q_k L)
\end{equation}

% Proton decay amplitude
\begin{equation}
\label{eq:proton-decay-amplitude}
\Gamma(p \to e^+ \pi^0) = \frac{m_p}{32\pi} |A|^2 \left(\frac{\alpha_{\text{GUT}}}{M_X}\right)^2
\end{equation}

% Proton lifetime formula (CRITICAL PREDICTION!)
\begin{equation}
\label{eq:proton-lifetime}
\tau_p = \Gamma^{-1} \approx \frac{M_X^4}{m_p^5} \times \frac{1}{\alpha_{\text{GUT}}^2}
\end{equation}

% Proton lifetime numerical value
\begin{equation}
\label{eq:proton-lifetime-value}
\tau_p \approx \frac{(2.118 \times 10^{16} \text{ GeV})^4}{(0.94 \text{ GeV})^5 \times (0.0425)^2} = 3.83 \times 10^{34} \text{ years}
\end{equation}


% ============================================
% SECTION 2.7: KK GRAVITON SPECTRUM
% ============================================

% KK graviton mass (first mode) (CRITICAL PREDICTION!)
\begin{equation}
\label{eq:kk-graviton-mass}
m_{\text{KK}}^{(1)} = 5.0 \pm 1.5 \text{ TeV}
\end{equation}

% KK mass tower
\begin{align}
m_{\text{KK}}^{(1)} &= 5.0 \text{ TeV} \label{eq:kk-m1} \\
m_{\text{KK}}^{(2)} &= 10.0 \text{ TeV} \label{eq:kk-m2} \\
m_{\text{KK}}^{(3)} &= 15.0 \text{ TeV} \label{eq:kk-m3}
\end{align}

% Discovery significance
\begin{equation}
\label{eq:kk-discovery}
\text{Discovery at HL-LHC: } 6.8\sigma \text{ with } 3000 \text{ fb}^{-1}
\end{equation}


% ============================================
% SECTION 2.8: PMNS MATRIX PREDICTIONS
% ============================================

% PMNS mixing angles (from G_2 geometry) (CRITICAL PREDICTIONS!)
\begin{align}
\theta_{12} &= 33.59^\circ \pm 0.24^\circ \label{eq:pmns-theta12} \\
\theta_{23} &= 47.20^\circ \pm 0.00^\circ \label{eq:pmns-theta23} \\
\theta_{13} &= 8.57^\circ \pm 0.01^\circ \label{eq:pmns-theta13} \\
\delta_{CP} &= 235^\circ \pm 0.1^\circ \label{eq:pmns-deltacp}
\end{align}

% PMNS maximal mixing deviation
\begin{equation}
\label{eq:pmns-deviation}
\theta_{23} - 45^\circ = 2.2^\circ \quad \text{(octant determination)}
\end{equation}

% Atmospheric mixing from torsion
\begin{equation}
\label{eq:theta23-from-torsion}
\alpha_4 + \alpha_5 = \frac{\ln(M_{\text{Pl}}/M_{\text{GUT}}) - T_\omega}{2\pi \times \text{flux}_{\text{norm}}}
\end{equation}

% Asymmetry parameter
\begin{equation}
\label{eq:theta23-asymmetry}
\alpha_4 - \alpha_5 = \frac{\Delta\theta_{23}}{n_{\text{gen}}} = \frac{2.2^\circ - 0^\circ}{3}
\end{equation}

% Torsion-derived α parameters
\begin{equation}
\label{eq:alpha-parameters}
\alpha_4 = \frac{(1.667 + 0.733)}{2} = 0.956, \quad \alpha_5 = \frac{(1.667 - 0.733)}{2} = 0.222
\end{equation}


% ============================================
% SECTION 2.9: BELL INEQUALITY AND CHSH
% ============================================

% CHSH inequality (entanglement test)
\begin{equation}
\label{eq:chsh-inequality}
|\langle \text{CHSH} \rangle| \leq 2 \text{ (local)} \quad \text{vs} \quad |\langle \text{CHSH} \rangle| \leq 2\sqrt{2} \text{ (quantum)}
\end{equation}


% ============================================
% SECTION 2.10: FRIEDMANN EQUATIONS
% ============================================

% Modified Friedmann equations (F(R,T,τ) gravity)
\begin{equation}
\label{eq:friedmann-1}
H^2 = \frac{8\pi G}{3} \rho_{\text{eff}} - \frac{k}{a^2} + \frac{\Lambda_{\text{eff}}}{3}
\end{equation}

\begin{equation}
\label{eq:friedmann-2}
\dot{H} = -4\pi G (\rho + p)_{\text{eff}} + \Gamma_\tau \tau
\end{equation}

% Effective cosmological constant
\begin{equation}
\label{eq:lambda-effective}
\Lambda_{\text{eff}}(a) = \Lambda_0 \left(1 + \epsilon \tau(a)\right)
\end{equation}


% ============================================
% CRITICAL VALUES SUMMARY
% ============================================

% Core predictions with numerical values
\begin{align}
\chi_{\text{eff}} &= 144 \quad \text{(flux quantization)} \label{eq:chi-eff-value} \\
n_{\text{gen}} &= 3 \quad \text{(from } \chi_{\text{eff}}/48\text{)} \label{eq:ngen-value} \\
M_{\text{GUT}} &= 2.118 \times 10^{16} \text{ GeV} \quad \text{(from } T_\omega\text{)} \label{eq:mgut-value} \\
w_0 &= -0.853 \quad \text{(from } D_{\text{eff}} = 12.577\text{)} \label{eq:w0-value} \\
\tau_p &= 3.83 \times 10^{34} \text{ years} \quad \text{(median)} \label{eq:taup-value} \\
\alpha_{\text{GUT}}^{-1} &= 23.54 \quad \text{(3-loop RG)} \label{eq:alphagut-value}
\end{align}

% ============================================
% END OF EQUATIONS
% ============================================
