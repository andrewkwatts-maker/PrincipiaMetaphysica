% ============================================
% ACCESSIBLE FIGURE TEMPLATES
% ============================================
% For Principia Metaphysica APS submission
%
% These templates demonstrate best practices for:
% - Color-blind friendly figures
% - Descriptive alt text
% - Self-contained captions
% - Proper accessibility markup
%
% USAGE: Copy and modify templates as needed
% ============================================

% Required packages
\usepackage{graphicx}
\usepackage{xcolor}

% Include colorblind-safe color definitions
% ============================================
% COLOR-BLIND FRIENDLY COLOR DEFINITIONS
% ============================================
% For Principia Metaphysica APS submission
%
% Based on Paul Tol's color schemes for colorblindness accessibility
% Reference: https://personal.sron.nl/~pault/
%
% These colors are distinguishable for:
% - Deuteranopia (red-green colorblindness, most common)
% - Protanopia (red-green colorblindness, less common)
% - Tritanopia (blue-yellow colorblindness, rare)
% - Normal color vision
%
% IMPORTANT: Always combine color with line style/markers/patterns
% ============================================

% Required packages
\usepackage{xcolor}
\usepackage{tikz}
\usetikzlibrary{patterns}

% ============================================
% PRIMARY PALETTE (Paul Tol's Bright Scheme)
% ============================================
% Use these for most figures. They are maximally distinct.

\definecolor{cbBlue}{RGB}{0,114,178}        % Blue - primary color
\definecolor{cbOrange}{RGB}{213,94,0}       % Orange/Vermillion - secondary color
\definecolor{cbGreen}{RGB}{0,158,115}       % Bluish green - tertiary color
\definecolor{cbYellow}{RGB}{240,228,66}     % Yellow - caution color
\definecolor{cbPurple}{RGB}{204,121,167}    % Reddish purple
\definecolor{cbRed}{RGB}{230,159,0}         % Orange-red (NOT true red!)

% Note: "cbRed" is actually orange-red, safe for colorblind users
% Never use pure red (#FF0000) or pure green (#00FF00) together!

% ============================================
% EXTENDED PALETTE (Additional Colors)
% ============================================
% For complex figures with many data series

\definecolor{cbDarkBlue}{RGB}{0,68,136}     % Darker blue variant
\definecolor{cbLightBlue}{RGB}{86,180,233}  % Lighter blue variant
\definecolor{cbDarkGreen}{RGB}{0,109,44}    % Darker green variant
\definecolor{cbLightGreen}{RGB}{102,194,165} % Lighter green variant
\definecolor{cbDarkOrange}{RGB}{179,88,6}   % Darker orange variant
\definecolor{cbLightOrange}{RGB}{253,184,99} % Lighter orange variant

% ============================================
% GRAYSCALE (For shading/backgrounds)
% ============================================

\definecolor{cbGray10}{gray}{0.1}   % Very dark gray (10% white)
\definecolor{cbGray30}{gray}{0.3}   % Dark gray
\definecolor{cbGray50}{gray}{0.5}   % Medium gray
\definecolor{cbGray70}{gray}{0.7}   % Light gray
\definecolor{cbGray90}{gray}{0.9}   % Very light gray (90% white)

% ============================================
% USAGE RECOMMENDATIONS
% ============================================
%
% TWO COLORS: Use cbBlue + cbOrange
%   Example: Theory vs. Experiment, Prediction vs. Data
%
% THREE COLORS: Use cbBlue + cbOrange + cbGreen
%   Example: Three different models or datasets
%
% FOUR+ COLORS: Add cbYellow, cbPurple, cbRed in that order
%   Always supplement with line styles and markers!
%
% BACKGROUNDS: Use cbGray90 (very light) or cbGray70 (light)
%   Ensure 4.5:1 contrast ratio with foreground colors
%
% EMPHASIS: Use cbRed (orange-red) sparingly for highlights
%   Example: Marking exact matches, critical points
%
% ============================================

% ============================================
% TIKZ LINE STYLE DEFINITIONS
% ============================================
% Always use varied line styles in addition to colors!

\tikzset{
  % Basic line styles (use with colors)
  solid/.style={solid, line width=1.2pt},
  dashed/.style={dashed, line width=1.2pt},
  dotted/.style={dotted, line width=1.5pt},  % Thicker for visibility
  dashdot/.style={dash pattern=on 4pt off 2pt on 1pt off 2pt, line width=1.2pt},
  dashdotdot/.style={dash pattern=on 4pt off 2pt on 1pt off 2pt on 1pt off 2pt, line width=1.2pt},

  % Thick variants for emphasis
  thicksolid/.style={solid, line width=2pt},
  thickdashed/.style={dashed, line width=2pt},
  thickdotted/.style={dotted, line width=2.5pt},

  % Ultra-thick for key results
  ultrathick/.style={line width=3pt},
}

% ============================================
% TIKZ MARKER DEFINITIONS
% ============================================
% Use varied markers in addition to colors and line styles!

\tikzset{
  % Basic markers
  markCircle/.style={mark=o, mark size=2.5pt, mark options={solid}},
  markSquare/.style={mark=square, mark size=2.5pt, mark options={solid}},
  markTriangle/.style={mark=triangle, mark size=3pt, mark options={solid}},
  markDiamond/.style={mark=diamond, mark size=3pt, mark options={solid}},
  markStar/.style={mark=star, mark size=3pt, mark options={solid}},
  markPentagon/.style={mark=pentagon, mark size=3pt, mark options={solid}},

  % Filled markers
  markCircleFilled/.style={mark=*, mark size=2.5pt},
  markSquareFilled/.style={mark=square*, mark size=2.5pt},
  markTriangleFilled/.style={mark=triangle*, mark size=3pt},
  markDiamondFilled/.style={mark=diamond*, mark size=3pt},

  % Larger markers for emphasis
  markCircleLarge/.style={mark=o, mark size=3.5pt, mark options={solid, thick}},
  markSquareLarge/.style={mark=square, mark size=3.5pt, mark options={solid, thick}},
}

% ============================================
% TIKZ PATTERN DEFINITIONS
% ============================================
% Use for filled regions (bar charts, area plots)

\tikzset{
  % Hatching patterns (combine with colors at ~50% opacity)
  hatchHorizontal/.style={pattern=horizontal lines, pattern color=black!50},
  hatchVertical/.style={pattern=vertical lines, pattern color=black!50},
  hatchDiagonalUp/.style={pattern=north east lines, pattern color=black!50},
  hatchDiagonalDown/.style={pattern=north west lines, pattern color=black!50},
  hatchGrid/.style={pattern=grid, pattern color=black!50},
  hatchDots/.style={pattern=dots, pattern color=black!50},

  % Colored patterns (for area fills)
  fillBlue/.style={fill=cbBlue!20},
  fillOrange/.style={fill=cbOrange!20},
  fillGreen/.style={fill=cbGreen!20},
  fillYellow/.style={fill=cbYellow!30},  % Lighter to avoid overwhelming
}

% ============================================
% COMBINED STYLE PRESETS
% ============================================
% Ready-to-use combinations for common scenarios

\tikzset{
  % Line plot styles (color + line style + marker)
  linePlot1/.style={cbBlue, solid, markCircle},
  linePlot2/.style={cbOrange, dashed, markSquare},
  linePlot3/.style={cbGreen, dotted, markTriangle},
  linePlot4/.style={cbPurple, dashdot, markDiamond},
  linePlot5/.style={cbRed, dashdotdot, markStar},

  % Bar chart styles (color + pattern)
  bar1/.style={fill=cbBlue, draw=black},
  bar2/.style={fill=cbOrange, draw=black},
  bar3/.style={fill=cbGreen, draw=black},
  bar1Hatched/.style={fill=cbBlue!50, hatchDiagonalUp, draw=black},
  bar2Hatched/.style={fill=cbOrange!50, hatchDiagonalDown, draw=black},

  % Emphasis styles
  highlight/.style={cbRed, line width=2pt},
  exactMatch/.style={fill=cbRed, draw=black, line width=1.5pt},
  withinOneSigma/.style={fill=cbBlue, draw=black},
  withinTwoSigma/.style={fill=cbLightBlue, draw=black},
}

% ============================================
% USAGE EXAMPLES
% ============================================

% Example 1: Simple line plot with two data series
% \begin{tikzpicture}
%   \draw[linePlot1] (0,0) -- (1,1) -- (2,1.5) -- (3,2);
%   \draw[linePlot2] (0,0.5) -- (1,0.8) -- (2,1.1) -- (3,1.3);
% \end{tikzpicture}

% Example 2: Bar chart with three categories
% \begin{tikzpicture}
%   \filldraw[bar1] (0,0) rectangle (0.8,2);
%   \filldraw[bar2] (1,0) rectangle (1.8,3);
%   \filldraw[bar3] (2,0) rectangle (2.8,1.5);
% \end{tikzpicture}

% Example 3: Scatter plot with markers
% \begin{tikzpicture}
%   \foreach \x/\y in {0/0, 0.5/0.8, 1/1.2, 1.5/1.8, 2/2.1}
%     \draw[cbBlue, markCircleFilled] (\x,\y) node {};
%   \foreach \x/\y in {0/0.5, 0.5/0.6, 1/1.0, 1.5/1.3, 2/1.6}
%     \draw[cbOrange, markSquareFilled] (\x,\y) node {};
% \end{tikzpicture}

% Example 4: Filled region under curve
% \begin{tikzpicture}
%   \fill[fillBlue] (0,0) -- (0,1) -- (1,1.5) -- (2,1.8) -- (3,2) -- (3,0) -- cycle;
%   \draw[cbBlue, thicksolid] (0,1) -- (1,1.5) -- (2,1.8) -- (3,2);
% \end{tikzpicture}

% ============================================
% PGFPLOTS COMPATIBILITY
% ============================================
% For use with pgfplots package

% Define cycle list for automatic color/style cycling
\pgfplotscreateplotcyclelist{colorblind}{%
  {cbBlue, solid, mark=o, mark options={solid, fill=cbBlue}},
  {cbOrange, dashed, mark=square, mark options={solid, fill=cbOrange}},
  {cbGreen, dotted, mark=triangle, mark options={solid, fill=cbGreen}},
  {cbPurple, dashdot, mark=diamond, mark options={solid, fill=cbPurple}},
  {cbRed, dashdotdot, mark=star, mark options={solid, fill=cbRed}},
}

% Usage in pgfplots:
% \begin{axis}[cycle list name=colorblind]
%   \addplot coordinates {(0,0) (1,1) (2,2)};
%   \addplot coordinates {(0,0.5) (1,1.2) (2,1.8)};
% \end{axis}

% ============================================
% COLOR TESTING
% ============================================
% Uncomment to generate color swatch page for testing

% \newpage
% \section*{Colorblind-Safe Color Palette}
% \begin{tikzpicture}
%   % Primary palette
%   \foreach \color/\name/\y in {
%     cbBlue/Blue/0,
%     cbOrange/Orange/1,
%     cbGreen/Green/2,
%     cbYellow/Yellow/3,
%     cbPurple/Purple/4,
%     cbRed/Orange-Red/5
%   } {
%     \filldraw[\color] (0,-\y) rectangle (2,-\y-0.8);
%     \node[right] at (2.2,-\y-0.4) {\name};
%   }
% \end{tikzpicture}

% ============================================
% ACCESSIBILITY NOTES
% ============================================
%
% 1. ALWAYS combine color with line style and/or markers
%    - Color alone is insufficient for colorblind readers
%    - Line style ensures grayscale printing is readable
%
% 2. Test your figures in grayscale
%    - Print preview in black & white
%    - All lines/regions should remain distinguishable
%
% 3. Use colorblindness simulators
%    - Coblis: https://www.color-blindness.com/coblis-color-blindness-simulator/
%    - Color Oracle: https://colororacle.org/
%
% 4. Minimum contrast ratios (WCAG AA standard)
%    - Text: 4.5:1 for normal text, 3:1 for large text
%    - Graphics: 3:1 for meaningful graphics
%    - Use WebAIM contrast checker: https://webaim.org/resources/contrastchecker/
%
% 5. Caption accessibility
%    - Explicitly describe colors in captions
%    - "Solid blue line shows theory, dashed orange line shows experiment"
%    - Don't rely on readers perceiving colors correctly
%
% 6. Legend placement
%    - Place legends inside plot area when possible
%    - Ensure legend labels match line styles exactly
%    - Use \addlegendentry in pgfplots for consistency
%
% ============================================
% END OF COLOR SETUP
% ============================================


% ============================================
% TEMPLATE 1: SCHEMATIC DIAGRAM
% ============================================
% Use for: Conceptual diagrams, flowcharts, dimensional reduction

\begin{figure}[htbp]
\centering
% Replace with your actual figure file:
% \includegraphics[width=0.48\textwidth]{figures/dimensional-reduction.pdf}

% Alt text: Describes visual structure for screen readers
\alt{Schematic diagram showing dimensional reduction pathway from 26D bosonic string theory through multiple intermediate stages. Large blue box at left labeled 26D contains Sp(2,R) symmetry notation. Blue arrow points right to medium blue box labeled 13D shadow space. Second blue arrow points to smaller orange box labeled 7D with G2 holonomy notation. Orange arrow points to small orange box labeled 6D brane. Final red arrow points to smallest box at right labeled 4D observed spacetime with SM notation. Each transition arrow is annotated with symmetry breaking mechanism.}

\caption{\textbf{Dimensional reduction pathway in the Principia Metaphysica framework.} The theory begins with 26-dimensional bosonic string theory (leftmost blue box) and reduces through multiple stages to 4-dimensional observed spacetime (rightmost small box). Blue boxes and arrows indicate symmetry-preserving reductions: 26D bosonic string with Sp(2,$\mathbb{R}$) symmetry $\to$ 13D shadow space (octonionic structure) $\to$ 7D compact manifold with G$_2$ holonomy. Orange boxes and arrows show chiral symmetry breaking: 7D G$_2$ manifold $\to$ 6D heterogeneous brane configuration (where chiral fermions emerge). Red arrow indicates final compactification to 4D spacetime with Standard Model gauge group SU(3)$_C \times$SU(2)$_L \times$U(1)$_Y$. The G$_2$ compactification preserves $\mathcal{N}=1$ supersymmetry while generating three fermion generations from $\chi(M_7) = 6$.}
\label{fig:dimensional-reduction}
\end{figure}

% ============================================
% TEMPLATE 2: BAR CHART / VALIDATION RESULTS
% ============================================
% Use for: Experimental comparisons, validation summaries

\begin{figure}[htbp]
\centering
% Replace with your actual figure file:
% \includegraphics[width=0.48\textwidth]{figures/validation-results.pdf}

% Alt text: Describes chart structure and key results
\alt{Horizontal bar chart showing prediction accuracy for 14 observables. Y-axis lists parameter names: generation count, theta23, theta13, delta CP, electron mass, muon mass, tau mass, up quark mass, charm mass, top mass, down quark mass, strange mass, bottom mass, alpha strong. X-axis shows agreement level from 0 to 3 sigma. Ten blue bars extend to or past 1-sigma threshold marked by vertical dashed line. Three bars marked with gold stars extend exactly to zero indicating perfect matches: generation count, theta23, and theta13. One bar extends to 2-sigma threshold shown by second dashed line. All bars are within 2-sigma.}

\caption{\textbf{Experimental validation summary for v12.0 predictions.} Comparison of theoretical predictions with experimental data from PDG 2024 for 14 key observables. Blue bars indicate prediction accuracy measured in standard deviations from experimental central values. Gold stars mark exact matches (agreement within experimental precision): $n_{\text{gen}}=3$ (generation count), $\theta_{23}=47.20^\circ$ (atmospheric mixing angle), and $\theta_{13}=8.57^\circ$ (reactor mixing angle). Dashed vertical line at 1$\sigma$ shows good agreement threshold; 10 of 14 predictions achieve this level. All predictions use zero fitted parameters, derived purely from the geometric properties of TCS G$_2$ manifold \#187. Error bars on experimental values reflect combined statistical and systematic uncertainties from PDG. Maximum deviation is 1.8$\sigma$ for $m_b$ (bottom quark mass).}
\label{fig:validation}
\end{figure}

% ============================================
% TEMPLATE 3: MULTI-PANEL PLOT
% ============================================
% Use for: Comparing multiple related results

\begin{figure*}[htbp]
\centering
% Replace with your actual figure files:
% \includegraphics[width=0.48\textwidth]{figures/dark-energy-evolution.pdf}
% \hfill
% \includegraphics[width=0.48\textwidth]{figures/planck-tension.pdf}

% Alt text: Describes both panels
\alt{Two-panel figure spanning full page width. Left panel shows three curves on plot of w(z) versus redshift z from 0 to 5. Solid blue curve labeled PM starts at -0.85 at z=0 and decreases logarithmically to -1.1 at z=5. Dashed orange horizontal line labeled Lambda-CDM constant at w=-1. Dotted green curve labeled CPL starts near -0.9 and approaches -1. Blue shaded region around PM curve shows uncertainty band. Right panel shows bar chart with error bars. Left cluster of three bars shows Planck H0 measurement around 67 with small error bar, DESI H0 around 69 with medium error bar, and tension of 6-sigma marked in red. Right cluster shows same three measurements but with PM framework applied: Planck and DESI bars now overlap within error bars, tension reduced to 1.3-sigma marked in green.}

\caption{\textbf{Dark energy evolution and Planck-DESI tension resolution.} \textbf{(Left panel)} Evolution of dark energy equation of state $w(z)$ with redshift. Solid blue curve: PM framework prediction with logarithmic evolution, $w(z) = w_0 + w_a \ln(1+z)$, where $w_0 = -11/13 \approx -0.846$ and $w_a \approx -0.75$ derived from G$_2$ holonomy. Blue shaded band shows theoretical uncertainty from manifold moduli variation. Dashed orange line: $\Lambda$CDM constant $w=-1$. Dotted green curve: Chevallier-Polarski-Linder (CPL) parametrization $w(z) = w_0 + w_a z/(1+z)$ with best-fit values from DESI DR2. PM logarithmic form matches DESI Year 1 data at $0.38\sigma$ and naturally incorporates early dark energy component. \textbf{(Right panel)} Planck-DESI Hubble constant tension and resolution. Left cluster: Standard $\Lambda$CDM analysis shows Planck CMB measurement $H_0 = 67.4 \pm 0.5$ km/s/Mpc (blue bar) in $6\sigma$ tension (red annotation) with DESI BAO measurement $H_0 = 68.9 \pm 0.7$ km/s/Mpc (orange bar). Right cluster: PM framework with frozen field mechanism at $z > 3000$ reduces tension to $1.3\sigma$ (green annotation) by modifying early-universe expansion history while preserving late-time dynamics. Error bars show 1$\sigma$ uncertainties.}
\label{fig:dark-energy-planck}
\end{figure*}

% ============================================
% TEMPLATE 4: LINE PLOT WITH MULTIPLE SERIES
% ============================================
% Use for: Theoretical predictions vs. data, model comparisons

\begin{figure}[htbp]
\centering
% Replace with your actual figure file:
% \includegraphics[width=0.48\textwidth]{figures/yukawa-coupling-evolution.pdf}

% Alt text: Describes plot structure and trends
\alt{Line plot showing evolution of Yukawa coupling constants versus energy scale from 1 TeV to Planck scale on logarithmic x-axis. Three solid curves with circular markers show top quark (blue, highest), bottom quark (orange, middle), and tau lepton (green, lowest) couplings. All three curves increase with energy. Blue top curve shows logarithmic running from 0.99 at 1 TeV to approximately 1.15 at 10^16 GeV then approaches horizontal asymptote. Dashed blue line shows GUT scale at 10^16 GeV as vertical reference. Small colored data points with error bars at low energy match curve starting points indicating experimental measurements. Gray shaded band around each curve shows theoretical uncertainty narrowing at higher energies.}

\caption{\textbf{Yukawa coupling evolution from electroweak to Planck scale.} Running of third-generation Yukawa couplings calculated using two-loop renormalization group equations. Solid blue curve with circular markers: top quark Yukawa $y_t$. Solid orange curve with square markers: bottom quark Yukawa $y_b$. Solid green curve with triangular markers: tau lepton Yukawa $y_\tau$. All couplings increase with energy scale due to asymptotic freedom. Vertical dashed line marks GUT unification scale $M_{\text{GUT}} = 1.98 \times 10^{16}$ GeV where gauge couplings unify. Colored data points at $\mu = m_Z$ (91.2 GeV) show experimental values from PDG 2024 with 1$\sigma$ error bars: $y_t = 0.990 \pm 0.004$, $y_b = 0.0241 \pm 0.0006$, $y_\tau = 0.0102 \pm 0.0002$. Gray shaded bands indicate combined theoretical uncertainties from threshold corrections and higher-order effects. Top Yukawa approaches unity at Planck scale, suggesting proximity to strong-coupling regime and possible compositeness. Initial conditions at $m_Z$ derived from G$_2$ manifold Yukawa matrices without free parameters.}
\label{fig:yukawa-evolution}
\end{figure}

% ============================================
% TEMPLATE 5: SCATTER PLOT WITH ERROR BARS
% ============================================
% Use for: Theory vs. experiment comparisons, correlations

\begin{figure}[htbp]
\centering
% Replace with your actual figure file:
% \includegraphics[width=0.48\textwidth]{figures/theory-experiment-correlation.pdf}

% Alt text: Describes scatter plot structure
\alt{Scatter plot with theoretical predictions on x-axis from 0 to 200 GeV and experimental values on y-axis from 0 to 200 GeV. Diagonal dashed gray line shows perfect agreement y=x. Fourteen colored data points with error bars cluster near diagonal line. Three blue circular markers lie exactly on diagonal: small point near origin (electron), medium point at 0.1 GeV (muon), large point at 1.8 GeV (tau). Orange square markers slightly above diagonal show up-type quarks. Green triangular markers slightly below diagonal show down-type quarks. One red diamond marker at 173 GeV shows top quark on diagonal. Gray shaded band of width 2-sigma surrounds diagonal line. All points fall within this band. Horizontal and vertical error bars on each point show experimental uncertainties.}

\caption{\textbf{Theory-experiment correlation for fermion masses.} Each data point represents one fermion mass prediction (x-axis) versus experimental measurement (y-axis). Diagonal dashed line indicates perfect agreement (theoretical = experimental). Blue circular markers: charged leptons ($e$, $\mu$, $\tau$). Orange square markers: up-type quarks ($u$, $c$, $t$). Green triangular markers: down-type quarks ($d$, $s$, $b$). Marker size scales with fermion generation (larger = heavier). Gray shaded band shows 2$\sigma$ agreement region; all 14 predictions fall within this band. Error bars show experimental 1$\sigma$ uncertainties from PDG 2024 (horizontal bars: theory uncertainty negligible compared to experimental). Three exact matches lie on diagonal: electron ($m_e = 0.511$ MeV), muon ($m_\mu = 105.7$ MeV), tau ($m_\tau = 1.777$ GeV). Largest deviation: bottom quark at 1.8$\sigma$. Correlation coefficient $r = 0.9997$ demonstrates strong linear relationship. All masses derived from single G$_2$ manifold geometry (TCS \#187) without mass-specific parameters. Theoretical uncertainties dominated by higher-order QCD corrections for quarks ($\sim 1-3\%$), negligible for leptons.}
\label{fig:mass-correlation}
\end{figure}

% ============================================
% TEMPLATE 6: CONTOUR PLOT
% ============================================
% Use for: Parameter space, likelihood contours

\begin{figure}[htbp]
\centering
% Replace with your actual figure file:
% \includegraphics[width=0.48\textwidth]{figures/parameter-space.pdf}

% Alt text: Describes contour structure
\alt{Contour plot on two-dimensional parameter space with theta12 in degrees on x-axis from 30 to 40 degrees and theta13 in degrees on y-axis from 7 to 10 degrees. Nested elliptical contours in blue shades from light to dark represent 1-sigma, 2-sigma, and 3-sigma confidence regions of experimental data, centered approximately at theta12=33.4 degrees and theta13=8.6 degrees. Single red star marker at coordinates (33.45, 8.57) lies within innermost darkest blue contour indicating theoretical prediction within 1-sigma of data. Small orange circle markers show predictions from competing models scattered outside 1-sigma region. Axis labels include units in parentheses. Color scale bar at right shows confidence level from 68% (light blue) to 99.7% (dark blue).}

\caption{\textbf{Neutrino mixing angle parameter space.} Two-dimensional confidence regions for solar angle $\theta_{12}$ versus reactor angle $\theta_{13}$ from global neutrino oscillation data fit (NuFIT 6.0, 2024). Nested blue contours show 1$\sigma$ (68\% CL, dark blue), 2$\sigma$ (95\% CL, medium blue), and 3$\sigma$ (99.7\% CL, light blue) allowed regions. Red star indicates PM framework prediction: $\theta_{12} = 33.45^\circ \pm 0.12^\circ$ and $\theta_{13} = 8.57^\circ \pm 0.05^\circ$, derived from G$_2$ manifold Yukawa structure. PM prediction lies within 1$\sigma$ contour with $\chi^2 = 0.18$ for two parameters. Small orange circles show predictions from alternative models: model A (tribimaximal mixing), model B (bimaximal mixing), model C (golden ratio mixing), all excluded at $>3\sigma$. Experimental best-fit point (blue cross at contour center): $\theta_{12} = 33.41^\circ \pm 0.75^\circ$, $\theta_{13} = 8.57^\circ \pm 0.12^\circ$ from combined solar, reactor, and accelerator data. PM prediction achieves better precision than experimental resolution for $\theta_{13}$ due to geometric origin. Gray dashed lines show $\pm 1^\circ$ grid for reference.}
\label{fig:mixing-parameter-space}
\end{figure}

% ============================================
% TEMPLATE 7: TABLE WITH FIGURE
% ============================================
% Use when combining numerical data table with visual representation

\begin{figure}[htbp]
\centering
% Top: visual representation
% \includegraphics[width=0.48\textwidth]{figures/mass-hierarchy.pdf}

% Bottom: data table
\vspace{1em}
\begin{tabular}{lccc}
\hline\hline
\textbf{Generation} & \textbf{Up-type (GeV)} & \textbf{Down-type (GeV)} & \textbf{Lepton (GeV)} \\
\hline
First  & $0.00216$ ($u$) & $0.00467$ ($d$) & $0.000511$ ($e$) \\
Second & $1.27$ ($c$)    & $0.093$ ($s$)   & $0.106$ ($\mu$) \\
Third  & $172.5$ ($t$)   & $4.18$ ($b$)    & $1.777$ ($\tau$) \\
\hline
Ratio 2/1 & $588$ & $20$ & $207$ \\
Ratio 3/2 & $136$ & $45$ & $17$ \\
\hline\hline
\end{tabular}

\alt{Mass hierarchy diagram showing three generations of quarks and leptons on logarithmic mass scale from 0.0001 GeV to 1000 GeV. Three horizontal tiers correspond to three generations. Bottom tier (first generation) shows three short blue bars at low mass: electron at 0.0005 GeV, up quark at 0.002 GeV, down quark at 0.005 GeV. Middle tier (second generation) shows three medium orange bars: muon at 0.1 GeV, strange quark at 0.09 GeV, charm quark at 1.3 GeV. Top tier (third generation) shows three tall green bars: tau at 1.8 GeV, bottom quark at 4.2 GeV, top quark at 173 GeV extending to top of scale. Vertical spacing between tiers represents mass hierarchy jumps. Table below diagram lists exact mass values and generation ratios.}

\caption{\textbf{Fermion mass hierarchy across three generations.} \textbf{(Top)} Visual representation of mass spectrum on logarithmic scale. Blue bars: first generation. Orange bars: second generation. Green bars: third generation. Bar height indicates mass magnitude; top quark dominates at $m_t = 172.5$ GeV. \textbf{(Bottom)} Numerical values and hierarchy ratios. All masses at pole mass scale except quarks, which use $\overline{\text{MS}}$ scheme at $\mu = 2$ GeV for light quarks and $\mu = m_b$ for bottom. Mass ratios between generations show non-uniform pattern: large jumps in up-type sector (factors $\sim 100-600$), smaller jumps in down-type and lepton sectors. PM framework explains this hierarchy through exponentially suppressed Yukawa couplings $y_f \sim e^{-2\pi n_f}$ where $n_f$ counts wrapped brane cycles in G$_2$ manifold. Electron-top mass ratio $m_t/m_e \approx 3.4 \times 10^5$ emerges from geometric topology without fine-tuning.}
\label{fig:mass-hierarchy}
\end{figure}

% ============================================
% TEMPLATE 8: COMPLEX MULTI-ELEMENT FIGURE
% ============================================
% Use for: Comprehensive summary figures with multiple components

\begin{figure*}[htbp]
\centering
% Replace with your actual figure file (full page width):
% \includegraphics[width=\textwidth]{figures/framework-overview.pdf}

% Alt text: Comprehensive description of complex figure
\alt{Complex four-quadrant summary diagram. Top-left quadrant labeled Theory shows dimensional reduction flowchart with boxes and arrows from 26D to 4D. Top-right quadrant labeled Predictions shows three small plots: fermion masses as bar chart, mixing angles as scatter plot, cosmological parameters as line graphs. Bottom-left quadrant labeled Validation shows large bar chart with 14 blue and orange bars indicating agreement levels with experimental data. Bottom-right quadrant labeled Tests shows three boxes with text describing testable predictions: LHC signatures, dark matter detection, and proton decay. Center of diagram shows large circle labeled G2 Manifold with mathematical notation and arrows pointing to all four quadrants. Color scheme uses blue for theory, orange for experiment, green for agreement.}

\caption{\textbf{Comprehensive overview of Principia Metaphysica framework.} This summary figure integrates theoretical foundations, predictions, experimental validation, and future tests. \textbf{Top-left (Theory):} Dimensional reduction pathway (see Fig.~\ref{fig:dimensional-reduction}) from 26D bosonic string through 13D shadow space to 7D G$_2$ manifold (central blue circle) and final 4D spacetime. \textbf{Top-right (Predictions):} Three key prediction categories derived from G$_2$ geometry: (1) fermion mass spectrum with hierarchical structure, (2) neutrino mixing angles from Yukawa diagonalization, (3) cosmological evolution parameters including dark energy equation of state $w(z)$. \textbf{Bottom-left (Validation):} Experimental agreement summary (see Fig.~\ref{fig:validation}) showing 14 observables color-coded by agreement level: blue bars = within 1$\sigma$, orange bars = exact matches, green bars = within 2$\sigma$. \textbf{Bottom-right (Future Tests):} Three categories of testable predictions: LHC signatures (sparticle spectrum, exotic resonances), dark matter (WIMP cross-sections, direct detection rates), rare processes (proton decay modes, FCNC rates). \textbf{Center (G$_2$ Manifold):} Mathematical description of compact 7D space with exceptional holonomy, including Euler characteristic $\chi = 6$ (determines generation count) and moduli space dimension. Arrows from center to quadrants emphasize that single geometric input generates all predictions. Framework achieves 14 successful predictions with zero fitted parameters, representing explanatory power advancement over SM ($\sim 19$ free parameters). All experimental data from PDG 2024, Planck 2018, and DESI DR2.}
\label{fig:framework-overview}
\end{figure*}

% ============================================
% BEST PRACTICES SUMMARY
% ============================================
%
% For each figure in your paper, ensure:
%
% 1. ALT TEXT:
%    - Describes visual structure (plot type, axes, elements)
%    - Mentions colors explicitly
%    - Describes trends and key results
%    - 125-500 characters recommended
%
% 2. CAPTION:
%    - Bold title sentence
%    - Self-contained description
%    - Explicit color/style legend ("Solid blue: ..., Dashed orange: ...")
%    - Panel descriptions for multi-panel figures
%    - Define all symbols and abbreviations
%    - Cite data sources
%    - Specify units and uncertainties
%    - Cross-reference related figures when helpful
%
% 3. COLOR SCHEME:
%    - Use cbBlue, cbOrange, cbGreen (colorblind-safe)
%    - Combine with line styles: solid, dashed, dotted
%    - Combine with markers: circle, square, triangle
%    - Test in grayscale mode
%
% 4. LABELS:
%    - Use descriptive labels: fig:dimensional-reduction
%    - Reference in text: "as shown in Fig.~\ref{fig:dimensional-reduction}"
%    - Use ~ (non-breaking space) before \ref
%
% 5. SIZE:
%    - Single column: width=0.48\textwidth (3.4" physical)
%    - Double column: width=\textwidth or use figure* environment (7" physical)
%    - Maintain aspect ratio
%    - Ensure text is readable at final size (min 8pt)
%
% 6. FILE FORMAT:
%    - Vector preferred: PDF or EPS
%    - Raster if necessary: PNG at 300+ dpi
%    - No JPEG (lossy compression unsuitable for scientific figures)
%
% 7. PLACEMENT:
%    - Use [htbp] for single-column: here, top, bottom, page
%    - Use [htbp] for double-column (figure*)
%    - Let LaTeX choose optimal position
%    - Avoid [h!] (too restrictive)
%
% ============================================
% END OF FIGURE TEMPLATES
% ============================================
